
% Default to the notebook output style

    


% Inherit from the specified cell style.




    
\documentclass{article}

    
    
    \usepackage{graphicx} % Used to insert images
    \usepackage{adjustbox} % Used to constrain images to a maximum size 
    \usepackage{color} % Allow colors to be defined
    \usepackage{enumerate} % Needed for markdown enumerations to work
    \usepackage{geometry} % Used to adjust the document margins
    \usepackage{amsmath} % Equations
    \usepackage{amssymb} % Equations
    \usepackage{eurosym} % defines \euro
    \usepackage[mathletters]{ucs} % Extended unicode (utf-8) support
    \usepackage[utf8x]{inputenc} % Allow utf-8 characters in the tex document
    \usepackage{fancyvrb} % verbatim replacement that allows latex
    \usepackage{grffile} % extends the file name processing of package graphics 
                         % to support a larger range 
    % The hyperref package gives us a pdf with properly built
    % internal navigation ('pdf bookmarks' for the table of contents,
    % internal cross-reference links, web links for URLs, etc.)
    \usepackage{hyperref}
    \usepackage{longtable} % longtable support required by pandoc >1.10
    \usepackage{booktabs}  % table support for pandoc > 1.12.2
    

    
    
    \definecolor{orange}{cmyk}{0,0.4,0.8,0.2}
    \definecolor{darkorange}{rgb}{.71,0.21,0.01}
    \definecolor{darkgreen}{rgb}{.12,.54,.11}
    \definecolor{myteal}{rgb}{.26, .44, .56}
    \definecolor{gray}{gray}{0.45}
    \definecolor{lightgray}{gray}{.95}
    \definecolor{mediumgray}{gray}{.8}
    \definecolor{inputbackground}{rgb}{.95, .95, .85}
    \definecolor{outputbackground}{rgb}{.95, .95, .95}
    \definecolor{traceback}{rgb}{1, .95, .95}
    % ansi colors
    \definecolor{red}{rgb}{.6,0,0}
    \definecolor{green}{rgb}{0,.65,0}
    \definecolor{brown}{rgb}{0.6,0.6,0}
    \definecolor{blue}{rgb}{0,.145,.698}
    \definecolor{purple}{rgb}{.698,.145,.698}
    \definecolor{cyan}{rgb}{0,.698,.698}
    \definecolor{lightgray}{gray}{0.5}
    
    % bright ansi colors
    \definecolor{darkgray}{gray}{0.25}
    \definecolor{lightred}{rgb}{1.0,0.39,0.28}
    \definecolor{lightgreen}{rgb}{0.48,0.99,0.0}
    \definecolor{lightblue}{rgb}{0.53,0.81,0.92}
    \definecolor{lightpurple}{rgb}{0.87,0.63,0.87}
    \definecolor{lightcyan}{rgb}{0.5,1.0,0.83}
    
    % commands and environments needed by pandoc snippets
    % extracted from the output of `pandoc -s`
    \DefineVerbatimEnvironment{Highlighting}{Verbatim}{commandchars=\\\{\}}
    % Add ',fontsize=\small' for more characters per line
    \newenvironment{Shaded}{}{}
    \newcommand{\KeywordTok}[1]{\textcolor[rgb]{0.00,0.44,0.13}{\textbf{{#1}}}}
    \newcommand{\DataTypeTok}[1]{\textcolor[rgb]{0.56,0.13,0.00}{{#1}}}
    \newcommand{\DecValTok}[1]{\textcolor[rgb]{0.25,0.63,0.44}{{#1}}}
    \newcommand{\BaseNTok}[1]{\textcolor[rgb]{0.25,0.63,0.44}{{#1}}}
    \newcommand{\FloatTok}[1]{\textcolor[rgb]{0.25,0.63,0.44}{{#1}}}
    \newcommand{\CharTok}[1]{\textcolor[rgb]{0.25,0.44,0.63}{{#1}}}
    \newcommand{\StringTok}[1]{\textcolor[rgb]{0.25,0.44,0.63}{{#1}}}
    \newcommand{\CommentTok}[1]{\textcolor[rgb]{0.38,0.63,0.69}{\textit{{#1}}}}
    \newcommand{\OtherTok}[1]{\textcolor[rgb]{0.00,0.44,0.13}{{#1}}}
    \newcommand{\AlertTok}[1]{\textcolor[rgb]{1.00,0.00,0.00}{\textbf{{#1}}}}
    \newcommand{\FunctionTok}[1]{\textcolor[rgb]{0.02,0.16,0.49}{{#1}}}
    \newcommand{\RegionMarkerTok}[1]{{#1}}
    \newcommand{\ErrorTok}[1]{\textcolor[rgb]{1.00,0.00,0.00}{\textbf{{#1}}}}
    \newcommand{\NormalTok}[1]{{#1}}
    
    % Define a nice break command that doesn't care if a line doesn't already
    % exist.
    \def\br{\hspace*{\fill} \\* }
    % Math Jax compatability definitions
    \def\gt{>}
    \def\lt{<}
    % Document parameters
    \title{Homework 2}
    \author{Roly Vicar\'ia \\ STAT 500, Summer 2015}
    
    

    % Pygments definitions
    
\makeatletter
\def\PY@reset{\let\PY@it=\relax \let\PY@bf=\relax%
    \let\PY@ul=\relax \let\PY@tc=\relax%
    \let\PY@bc=\relax \let\PY@ff=\relax}
\def\PY@tok#1{\csname PY@tok@#1\endcsname}
\def\PY@toks#1+{\ifx\relax#1\empty\else%
    \PY@tok{#1}\expandafter\PY@toks\fi}
\def\PY@do#1{\PY@bc{\PY@tc{\PY@ul{%
    \PY@it{\PY@bf{\PY@ff{#1}}}}}}}
\def\PY#1#2{\PY@reset\PY@toks#1+\relax+\PY@do{#2}}

\expandafter\def\csname PY@tok@gd\endcsname{\def\PY@tc##1{\textcolor[rgb]{0.63,0.00,0.00}{##1}}}
\expandafter\def\csname PY@tok@gu\endcsname{\let\PY@bf=\textbf\def\PY@tc##1{\textcolor[rgb]{0.50,0.00,0.50}{##1}}}
\expandafter\def\csname PY@tok@gt\endcsname{\def\PY@tc##1{\textcolor[rgb]{0.00,0.27,0.87}{##1}}}
\expandafter\def\csname PY@tok@gs\endcsname{\let\PY@bf=\textbf}
\expandafter\def\csname PY@tok@gr\endcsname{\def\PY@tc##1{\textcolor[rgb]{1.00,0.00,0.00}{##1}}}
\expandafter\def\csname PY@tok@cm\endcsname{\let\PY@it=\textit\def\PY@tc##1{\textcolor[rgb]{0.25,0.50,0.50}{##1}}}
\expandafter\def\csname PY@tok@vg\endcsname{\def\PY@tc##1{\textcolor[rgb]{0.10,0.09,0.49}{##1}}}
\expandafter\def\csname PY@tok@m\endcsname{\def\PY@tc##1{\textcolor[rgb]{0.40,0.40,0.40}{##1}}}
\expandafter\def\csname PY@tok@mh\endcsname{\def\PY@tc##1{\textcolor[rgb]{0.40,0.40,0.40}{##1}}}
\expandafter\def\csname PY@tok@go\endcsname{\def\PY@tc##1{\textcolor[rgb]{0.53,0.53,0.53}{##1}}}
\expandafter\def\csname PY@tok@ge\endcsname{\let\PY@it=\textit}
\expandafter\def\csname PY@tok@vc\endcsname{\def\PY@tc##1{\textcolor[rgb]{0.10,0.09,0.49}{##1}}}
\expandafter\def\csname PY@tok@il\endcsname{\def\PY@tc##1{\textcolor[rgb]{0.40,0.40,0.40}{##1}}}
\expandafter\def\csname PY@tok@cs\endcsname{\let\PY@it=\textit\def\PY@tc##1{\textcolor[rgb]{0.25,0.50,0.50}{##1}}}
\expandafter\def\csname PY@tok@cp\endcsname{\def\PY@tc##1{\textcolor[rgb]{0.74,0.48,0.00}{##1}}}
\expandafter\def\csname PY@tok@gi\endcsname{\def\PY@tc##1{\textcolor[rgb]{0.00,0.63,0.00}{##1}}}
\expandafter\def\csname PY@tok@gh\endcsname{\let\PY@bf=\textbf\def\PY@tc##1{\textcolor[rgb]{0.00,0.00,0.50}{##1}}}
\expandafter\def\csname PY@tok@ni\endcsname{\let\PY@bf=\textbf\def\PY@tc##1{\textcolor[rgb]{0.60,0.60,0.60}{##1}}}
\expandafter\def\csname PY@tok@nl\endcsname{\def\PY@tc##1{\textcolor[rgb]{0.63,0.63,0.00}{##1}}}
\expandafter\def\csname PY@tok@nn\endcsname{\let\PY@bf=\textbf\def\PY@tc##1{\textcolor[rgb]{0.00,0.00,1.00}{##1}}}
\expandafter\def\csname PY@tok@no\endcsname{\def\PY@tc##1{\textcolor[rgb]{0.53,0.00,0.00}{##1}}}
\expandafter\def\csname PY@tok@na\endcsname{\def\PY@tc##1{\textcolor[rgb]{0.49,0.56,0.16}{##1}}}
\expandafter\def\csname PY@tok@nb\endcsname{\def\PY@tc##1{\textcolor[rgb]{0.00,0.50,0.00}{##1}}}
\expandafter\def\csname PY@tok@nc\endcsname{\let\PY@bf=\textbf\def\PY@tc##1{\textcolor[rgb]{0.00,0.00,1.00}{##1}}}
\expandafter\def\csname PY@tok@nd\endcsname{\def\PY@tc##1{\textcolor[rgb]{0.67,0.13,1.00}{##1}}}
\expandafter\def\csname PY@tok@ne\endcsname{\let\PY@bf=\textbf\def\PY@tc##1{\textcolor[rgb]{0.82,0.25,0.23}{##1}}}
\expandafter\def\csname PY@tok@nf\endcsname{\def\PY@tc##1{\textcolor[rgb]{0.00,0.00,1.00}{##1}}}
\expandafter\def\csname PY@tok@si\endcsname{\let\PY@bf=\textbf\def\PY@tc##1{\textcolor[rgb]{0.73,0.40,0.53}{##1}}}
\expandafter\def\csname PY@tok@s2\endcsname{\def\PY@tc##1{\textcolor[rgb]{0.73,0.13,0.13}{##1}}}
\expandafter\def\csname PY@tok@vi\endcsname{\def\PY@tc##1{\textcolor[rgb]{0.10,0.09,0.49}{##1}}}
\expandafter\def\csname PY@tok@nt\endcsname{\let\PY@bf=\textbf\def\PY@tc##1{\textcolor[rgb]{0.00,0.50,0.00}{##1}}}
\expandafter\def\csname PY@tok@nv\endcsname{\def\PY@tc##1{\textcolor[rgb]{0.10,0.09,0.49}{##1}}}
\expandafter\def\csname PY@tok@s1\endcsname{\def\PY@tc##1{\textcolor[rgb]{0.73,0.13,0.13}{##1}}}
\expandafter\def\csname PY@tok@kd\endcsname{\let\PY@bf=\textbf\def\PY@tc##1{\textcolor[rgb]{0.00,0.50,0.00}{##1}}}
\expandafter\def\csname PY@tok@sh\endcsname{\def\PY@tc##1{\textcolor[rgb]{0.73,0.13,0.13}{##1}}}
\expandafter\def\csname PY@tok@sc\endcsname{\def\PY@tc##1{\textcolor[rgb]{0.73,0.13,0.13}{##1}}}
\expandafter\def\csname PY@tok@sx\endcsname{\def\PY@tc##1{\textcolor[rgb]{0.00,0.50,0.00}{##1}}}
\expandafter\def\csname PY@tok@bp\endcsname{\def\PY@tc##1{\textcolor[rgb]{0.00,0.50,0.00}{##1}}}
\expandafter\def\csname PY@tok@c1\endcsname{\let\PY@it=\textit\def\PY@tc##1{\textcolor[rgb]{0.25,0.50,0.50}{##1}}}
\expandafter\def\csname PY@tok@kc\endcsname{\let\PY@bf=\textbf\def\PY@tc##1{\textcolor[rgb]{0.00,0.50,0.00}{##1}}}
\expandafter\def\csname PY@tok@c\endcsname{\let\PY@it=\textit\def\PY@tc##1{\textcolor[rgb]{0.25,0.50,0.50}{##1}}}
\expandafter\def\csname PY@tok@mf\endcsname{\def\PY@tc##1{\textcolor[rgb]{0.40,0.40,0.40}{##1}}}
\expandafter\def\csname PY@tok@err\endcsname{\def\PY@bc##1{\setlength{\fboxsep}{0pt}\fcolorbox[rgb]{1.00,0.00,0.00}{1,1,1}{\strut ##1}}}
\expandafter\def\csname PY@tok@mb\endcsname{\def\PY@tc##1{\textcolor[rgb]{0.40,0.40,0.40}{##1}}}
\expandafter\def\csname PY@tok@ss\endcsname{\def\PY@tc##1{\textcolor[rgb]{0.10,0.09,0.49}{##1}}}
\expandafter\def\csname PY@tok@sr\endcsname{\def\PY@tc##1{\textcolor[rgb]{0.73,0.40,0.53}{##1}}}
\expandafter\def\csname PY@tok@mo\endcsname{\def\PY@tc##1{\textcolor[rgb]{0.40,0.40,0.40}{##1}}}
\expandafter\def\csname PY@tok@kn\endcsname{\let\PY@bf=\textbf\def\PY@tc##1{\textcolor[rgb]{0.00,0.50,0.00}{##1}}}
\expandafter\def\csname PY@tok@mi\endcsname{\def\PY@tc##1{\textcolor[rgb]{0.40,0.40,0.40}{##1}}}
\expandafter\def\csname PY@tok@gp\endcsname{\let\PY@bf=\textbf\def\PY@tc##1{\textcolor[rgb]{0.00,0.00,0.50}{##1}}}
\expandafter\def\csname PY@tok@o\endcsname{\def\PY@tc##1{\textcolor[rgb]{0.40,0.40,0.40}{##1}}}
\expandafter\def\csname PY@tok@kr\endcsname{\let\PY@bf=\textbf\def\PY@tc##1{\textcolor[rgb]{0.00,0.50,0.00}{##1}}}
\expandafter\def\csname PY@tok@s\endcsname{\def\PY@tc##1{\textcolor[rgb]{0.73,0.13,0.13}{##1}}}
\expandafter\def\csname PY@tok@kp\endcsname{\def\PY@tc##1{\textcolor[rgb]{0.00,0.50,0.00}{##1}}}
\expandafter\def\csname PY@tok@w\endcsname{\def\PY@tc##1{\textcolor[rgb]{0.73,0.73,0.73}{##1}}}
\expandafter\def\csname PY@tok@kt\endcsname{\def\PY@tc##1{\textcolor[rgb]{0.69,0.00,0.25}{##1}}}
\expandafter\def\csname PY@tok@ow\endcsname{\let\PY@bf=\textbf\def\PY@tc##1{\textcolor[rgb]{0.67,0.13,1.00}{##1}}}
\expandafter\def\csname PY@tok@sb\endcsname{\def\PY@tc##1{\textcolor[rgb]{0.73,0.13,0.13}{##1}}}
\expandafter\def\csname PY@tok@k\endcsname{\let\PY@bf=\textbf\def\PY@tc##1{\textcolor[rgb]{0.00,0.50,0.00}{##1}}}
\expandafter\def\csname PY@tok@se\endcsname{\let\PY@bf=\textbf\def\PY@tc##1{\textcolor[rgb]{0.73,0.40,0.13}{##1}}}
\expandafter\def\csname PY@tok@sd\endcsname{\let\PY@it=\textit\def\PY@tc##1{\textcolor[rgb]{0.73,0.13,0.13}{##1}}}

\def\PYZbs{\char`\\}
\def\PYZus{\char`\_}
\def\PYZob{\char`\{}
\def\PYZcb{\char`\}}
\def\PYZca{\char`\^}
\def\PYZam{\char`\&}
\def\PYZlt{\char`\<}
\def\PYZgt{\char`\>}
\def\PYZsh{\char`\#}
\def\PYZpc{\char`\%}
\def\PYZdl{\char`\$}
\def\PYZhy{\char`\-}
\def\PYZsq{\char`\'}
\def\PYZdq{\char`\"}
\def\PYZti{\char`\~}
% for compatibility with earlier versions
\def\PYZat{@}
\def\PYZlb{[}
\def\PYZrb{]}
\makeatother


    % Exact colors from NB
    \definecolor{incolor}{rgb}{0.0, 0.0, 0.5}
    \definecolor{outcolor}{rgb}{0.545, 0.0, 0.0}



    
    % Prevent overflowing lines due to hard-to-break entities
    \sloppy 
    % Setup hyperref package
    \hypersetup{
      breaklinks=true,  % so long urls are correctly broken across lines
      colorlinks=true,
      urlcolor=blue,
      linkcolor=darkorange,
      citecolor=darkgreen,
      }
    % Slightly bigger margins than the latex defaults
    
    \geometry{verbose,tmargin=1in,bmargin=1in,lmargin=1in,rmargin=1in}
    
    

    \begin{document}
    
    
    \maketitle  

    
    \begin{center}\rule{3in}{0.4pt}\end{center}

    \subsubsection{Question 1}\label{question-1}

    \begin{Verbatim}[commandchars=\\\{\}]
{\color{incolor}In [{\color{incolor}1}]:} \PY{k+kn}{import} \PY{n+nn}{numpy} \PY{k+kn}{as} \PY{n+nn}{np}
        \PY{k+kn}{import} \PY{n+nn}{pandas} \PY{k+kn}{as} \PY{n+nn}{pd}
        
        \PY{c}{\PYZsh{} DDE to PCB ratios}
        \PY{n}{terrestrial} \PY{o}{=} \PY{n}{pd}\PY{o}{.}\PY{n}{Series}\PY{p}{(}\PY{p}{[}\PY{l+m+mf}{76.5}\PY{p}{,}\PY{l+m+mf}{6.03}\PY{p}{,}\PY{l+m+mf}{3.51}\PY{p}{,}\PY{l+m+mf}{9.96}\PY{p}{,}\PY{l+m+mf}{4.24}\PY{p}{,}\PY{l+m+mf}{7.74}\PY{p}{,}\PY{l+m+mf}{9.54}\PY{p}{,}\PY{l+m+mf}{41.7}\PY{p}{,}\PY{l+m+mf}{1.84}\PY{p}{,}\PY{l+m+mf}{2.5}\PY{p}{,}\PY{l+m+mf}{1.64}\PY{p}{]}\PY{p}{)}
        \PY{n}{aquatic} \PY{o}{=} \PY{n}{pd}\PY{o}{.}\PY{n}{Series}\PY{p}{(}\PY{p}{[}\PY{o}{.}\PY{l+m+mi}{27}\PY{p}{,}\PY{o}{.}\PY{l+m+mi}{61}\PY{p}{,}\PY{o}{.}\PY{l+m+mi}{54}\PY{p}{,}\PY{o}{.}\PY{l+m+mi}{14}\PY{p}{,}\PY{o}{.}\PY{l+m+mi}{63}\PY{p}{,}\PY{o}{.}\PY{l+m+mi}{23}\PY{p}{,}\PY{o}{.}\PY{l+m+mi}{56}\PY{p}{,}\PY{o}{.}\PY{l+m+mi}{48}\PY{p}{,}\PY{o}{.}\PY{l+m+mi}{16}\PY{p}{,}\PY{o}{.}\PY{l+m+mi}{18}\PY{p}{]}\PY{p}{)}
        
        \PY{n}{df} \PY{o}{=} \PY{n}{pd}\PY{o}{.}\PY{n}{DataFrame}\PY{p}{(}\PY{p}{\PYZob{}}\PY{l+s}{\PYZsq{}}\PY{l+s}{terrestrial}\PY{l+s}{\PYZsq{}} \PY{p}{:} \PY{n}{terrestrial}\PY{p}{,} \PY{l+s}{\PYZsq{}}\PY{l+s}{aquatic}\PY{l+s}{\PYZsq{}} \PY{p}{:} \PY{n}{aquatic}\PY{p}{\PYZcb{}}\PY{p}{)}
\end{Verbatim}

    \paragraph{1a. Compute mean and median for 21 data points, ignoring type
of
feeder}\label{a.-compute-mean-and-median-for-21-data-points-ignoring-type-of-feeder}

    \begin{Verbatim}[commandchars=\\\{\}]
{\color{incolor}In [{\color{incolor}2}]:} \PY{n}{combined} \PY{o}{=} \PY{n}{pd}\PY{o}{.}\PY{n}{concat}\PY{p}{(}\PY{p}{[}\PY{n}{df}\PY{p}{[}\PY{l+s}{\PYZdq{}}\PY{l+s}{aquatic}\PY{l+s}{\PYZdq{}}\PY{p}{]}\PY{p}{,} \PY{n}{df}\PY{p}{[}\PY{l+s}{\PYZdq{}}\PY{l+s}{terrestrial}\PY{l+s}{\PYZdq{}}\PY{p}{]}\PY{p}{]}\PY{p}{)}
        \PY{k}{print} \PY{l+s}{\PYZdq{}}\PY{l+s}{Combined mean: }\PY{l+s+si}{\PYZpc{}s}\PY{l+s}{\PYZdq{}} \PY{o}{\PYZpc{}} \PY{n}{combined}\PY{o}{.}\PY{n}{mean}\PY{p}{(}\PY{p}{)}
        \PY{k}{print} \PY{l+s}{\PYZdq{}}\PY{l+s}{Combined median: }\PY{l+s+si}{\PYZpc{}s}\PY{l+s}{\PYZdq{}} \PY{o}{\PYZpc{}} \PY{n}{combined}\PY{o}{.}\PY{n}{median}\PY{p}{(}\PY{p}{)}
\end{Verbatim}

    \begin{Verbatim}[commandchars=\\\{\}]
Combined mean: 8.04761904762
Combined median: 1.64
    \end{Verbatim}

    \paragraph{1b. Compute mean and median separately for each
type}\label{b.-compute-mean-and-median-separately-for-each-type}

    \begin{Verbatim}[commandchars=\\\{\}]
{\color{incolor}In [{\color{incolor}3}]:} \PY{k}{print} \PY{l+s}{\PYZdq{}}\PY{l+s}{Aquatic mean: }\PY{l+s+si}{\PYZpc{}s}\PY{l+s}{\PYZdq{}} \PY{o}{\PYZpc{}} \PY{n}{df}\PY{p}{[}\PY{l+s}{\PYZdq{}}\PY{l+s}{aquatic}\PY{l+s}{\PYZdq{}}\PY{p}{]}\PY{o}{.}\PY{n}{mean}\PY{p}{(}\PY{p}{)}
        \PY{k}{print} \PY{l+s}{\PYZdq{}}\PY{l+s}{Aquatic median: }\PY{l+s+si}{\PYZpc{}s}\PY{l+s}{\PYZdq{}} \PY{o}{\PYZpc{}} \PY{n}{df}\PY{p}{[}\PY{l+s}{\PYZdq{}}\PY{l+s}{aquatic}\PY{l+s}{\PYZdq{}}\PY{p}{]}\PY{o}{.}\PY{n}{median}\PY{p}{(}\PY{p}{)}
        
        \PY{k}{print} \PY{l+s}{\PYZdq{}}\PY{l+s}{Terrestrial mean: }\PY{l+s+si}{\PYZpc{}s}\PY{l+s}{\PYZdq{}} \PY{o}{\PYZpc{}} \PY{n}{df}\PY{p}{[}\PY{l+s}{\PYZdq{}}\PY{l+s}{terrestrial}\PY{l+s}{\PYZdq{}}\PY{p}{]}\PY{o}{.}\PY{n}{mean}\PY{p}{(}\PY{p}{)}
        \PY{k}{print} \PY{l+s}{\PYZdq{}}\PY{l+s}{Terrestrial median: }\PY{l+s+si}{\PYZpc{}s}\PY{l+s}{\PYZdq{}} \PY{o}{\PYZpc{}} \PY{n}{df}\PY{p}{[}\PY{l+s}{\PYZdq{}}\PY{l+s}{terrestrial}\PY{l+s}{\PYZdq{}}\PY{p}{]}\PY{o}{.}\PY{n}{median}\PY{p}{(}\PY{p}{)}
\end{Verbatim}

    \begin{Verbatim}[commandchars=\\\{\}]
Aquatic mean: 0.38
Aquatic median: 0.375
Terrestrial mean: 15.0181818182
Terrestrial median: 6.03
    \end{Verbatim}

    \paragraph{1c. Comment on the relative sensitivity of the mean and
median to extreme values in a data
set}\label{c.-comment-on-the-relative-sensitivity-of-the-mean-and-median-to-extreme-values-in-a-data-set}

    \begin{Verbatim}[commandchars=\\\{\}]
{\color{incolor}In [{\color{incolor}4}]:} \PY{c}{\PYZsh{} trimming out 2 large values from terrestrial group}
        \PY{n}{df}\PY{o}{.}\PY{n}{terrestrial}\PY{p}{[}\PY{n}{df}\PY{o}{.}\PY{n}{terrestrial} \PY{o}{\PYZlt{}} \PY{l+m+mi}{40}\PY{p}{]}\PY{o}{.}\PY{n}{mean}\PY{p}{(}\PY{p}{)}
\end{Verbatim}

            \begin{Verbatim}[commandchars=\\\{\}]
{\color{outcolor}Out[{\color{outcolor}4}]:} 5.2222222222222223
\end{Verbatim}
        
    The results show that the mean is more sensitive to extreme values than
the median. I think it's most apparent when looking at the terrestrial
values where the mean is \textasciitilde{}15 and the median is
\textasciitilde{}6. Two large values sway the mean. If we took the mean
without those two large values, the mean of the terrestrial data would
be 5.22.

    \paragraph{1d. Comment on the appropriateness of using mean and median
as measure of central location for DDE to PCB ratio for two types of
feeders.}\label{d.-comment-on-the-appropriateness-of-using-mean-and-median-as-measure-of-central-location-for-dde-to-pcb-ratio-for-two-types-of-feeders.}

    I think that for the terrestrial group, the appropriate measure would be
the median since the mean was swayed significantly by two larger values.
For the aquatic group, either one would work since they are similar and
the data is fairly evenly distributed.

    \paragraph{1e. Compute a measure of dispersion for the two types of
feeder separately that is
resistant}\label{e.-compute-a-measure-of-dispersion-for-the-two-types-of-feeder-separately-that-is-resistant}

    \begin{Verbatim}[commandchars=\\\{\}]
{\color{incolor}In [{\color{incolor}5}]:} \PY{k+kn}{import} \PY{n+nn}{mycode} \PY{k+kn}{as} \PY{n+nn}{m}
        \PY{c}{\PYZsh{}importing mycode because Minitab\PYZsq{}s algorithm for quartiles does not }
        \PY{c}{\PYZsh{}exist in python. Had to implement myself}
        
        \PY{n}{q\PYZus{}low} \PY{o}{=} \PY{n}{m}\PY{o}{.}\PY{n}{quartile}\PY{p}{(}\PY{n}{df}\PY{o}{.}\PY{n}{aquatic}\PY{p}{,} \PY{l+m+mi}{1}\PY{p}{)}
        \PY{n}{q\PYZus{}high} \PY{o}{=} \PY{n}{m}\PY{o}{.}\PY{n}{quartile}\PY{p}{(}\PY{n}{df}\PY{o}{.}\PY{n}{aquatic}\PY{p}{,} \PY{l+m+mi}{3}\PY{p}{)}
        \PY{n}{aquatic\PYZus{}iqr} \PY{o}{=} \PY{n}{q\PYZus{}high} \PY{o}{\PYZhy{}} \PY{n}{q\PYZus{}low}
        
        \PY{n}{q\PYZus{}low} \PY{o}{=} \PY{n}{m}\PY{o}{.}\PY{n}{quartile}\PY{p}{(}\PY{n}{df}\PY{o}{.}\PY{n}{terrestrial}\PY{p}{,} \PY{l+m+mi}{1}\PY{p}{)}
        \PY{n}{q\PYZus{}high} \PY{o}{=} \PY{n}{m}\PY{o}{.}\PY{n}{quartile}\PY{p}{(}\PY{n}{df}\PY{o}{.}\PY{n}{terrestrial}\PY{p}{,} \PY{l+m+mi}{3}\PY{p}{)}
        \PY{n}{terrestrial\PYZus{}iqr} \PY{o}{=} \PY{n}{q\PYZus{}high} \PY{o}{\PYZhy{}} \PY{n}{q\PYZus{}low}
        
        \PY{k}{print} \PY{l+s}{\PYZdq{}}\PY{l+s}{Aquatic IQR: }\PY{l+s+si}{\PYZpc{}s}\PY{l+s}{\PYZdq{}} \PY{o}{\PYZpc{}} \PY{n}{aquatic\PYZus{}iqr}
        \PY{k}{print} \PY{l+s}{\PYZdq{}}\PY{l+s}{Terrestrial IQR: }\PY{l+s+si}{\PYZpc{}s}\PY{l+s}{\PYZdq{}} \PY{o}{\PYZpc{}} \PY{n}{terrestrial\PYZus{}iqr}
\end{Verbatim}

    \begin{Verbatim}[commandchars=\\\{\}]
Aquatic IQR: 0.3975
Terrestrial IQR: 7.46
    \end{Verbatim}

    \paragraph{1f. Compute a measure of dispersion for the two types of
feeder separately that is non
resistant}\label{f.-compute-a-measure-of-dispersion-for-the-two-types-of-feeder-separately-that-is-non-resistant}

    \begin{Verbatim}[commandchars=\\\{\}]
{\color{incolor}In [{\color{incolor}6}]:} \PY{k}{print} \PY{l+s}{\PYZdq{}}\PY{l+s}{Aquatic range: }\PY{l+s+si}{\PYZpc{}s}\PY{l+s}{\PYZdq{}} \PY{o}{\PYZpc{}} \PY{p}{(}\PY{n}{df}\PY{o}{.}\PY{n}{aquatic}\PY{o}{.}\PY{n}{max}\PY{p}{(}\PY{p}{)} \PY{o}{\PYZhy{}} \PY{n}{df}\PY{o}{.}\PY{n}{aquatic}\PY{o}{.}\PY{n}{min}\PY{p}{(}\PY{p}{)}\PY{p}{)}
        \PY{k}{print} \PY{l+s}{\PYZdq{}}\PY{l+s}{Terrestrial range: }\PY{l+s+si}{\PYZpc{}s}\PY{l+s}{\PYZdq{}} \PY{o}{\PYZpc{}} \PY{p}{(}\PY{n}{df}\PY{o}{.}\PY{n}{terrestrial}\PY{o}{.}\PY{n}{max}\PY{p}{(}\PY{p}{)} \PY{o}{\PYZhy{}} \PY{n}{df}\PY{o}{.}\PY{n}{terrestrial}\PY{o}{.}\PY{n}{min}\PY{p}{(}\PY{p}{)}\PY{p}{)}
\end{Verbatim}

    \begin{Verbatim}[commandchars=\\\{\}]
Aquatic range: 0.49
Terrestrial range: 74.86
    \end{Verbatim}

    \begin{center}\rule{3in}{0.4pt}\end{center}

\subsubsection{Question 2}\label{question-2}

    \paragraph{2a. When the data set has some extreme outliers, can we trim
some of the data and use the trimmed data set to carry out analysis to
make inference about the central
location?}\label{a.-when-the-data-set-has-some-extreme-outliers-can-we-trim-some-of-the-data-and-use-the-trimmed-data-set-to-carry-out-analysis-to-make-inference-about-the-central-location}

    Section 3.4 of the book says that by trimming the data we can get a more
reliable measure of the central value. It then goes on to say that this
is particularly important when the sample mean is used to predict the
population central value. So the answer to the question is yes, trimming
some of the data can be used to make inference about the central
location. I think there are some caveats to this and one needs to be
mindful about the data they are excluding. I think it's very possible
that your data might be bimodal and you may be discarding evidence of
that second mound if you don't have enough samples.

    \paragraph{2b. When the data set has some extreme outliers, can we trim
some of the data and use the trimmed data set to carry out analysis to
make inference about the dispersion of the
population?}\label{b.-when-the-data-set-has-some-extreme-outliers-can-we-trim-some-of-the-data-and-use-the-trimmed-data-set-to-carry-out-analysis-to-make-inference-about-the-dispersion-of-the-population}

    I think it's possible to make meaningful observations about the
dispersion of the data using a trimmed data set. The boxplot is a great
way to visualize the variability of the data and it identifies the
potential outliers that are outside of the lower/upper limits.

I want to distinguish between ``observations'' and ``carrying out
analysis/make inferences''. What I stated above is that using a boxplot,
you can exclude outliers and still make meaningul observations. But I
think that by trimming data, which significantly alters the standard
deviation, you can't make any inferences about the population.

I think the main difference between using trimmed data for inferring
central location versus using trimmed data for inferring variability is
that central location has other measures that you can use to validate
the trimmed mean. You can compare it to the original median or mode to
validate the impact of the trimming. But you don't have that safety net
when using trimmed data for analyzing variability.

    \begin{center}\rule{3in}{0.4pt}\end{center}

\subsubsection{Question 3}\label{question-3}

    \begin{Verbatim}[commandchars=\\\{\}]
{\color{incolor}In [{\color{incolor}7}]:} \PY{n}{supplier\PYZus{}1} \PY{o}{=} \PY{n}{pd}\PY{o}{.}\PY{n}{Series}\PY{p}{(}\PY{p}{[}\PY{l+m+mf}{189.9}\PY{p}{,}\PY{l+m+mf}{191.9}\PY{p}{,}\PY{l+m+mf}{190.9}\PY{p}{,}\PY{l+m+mf}{183.8}\PY{p}{,}\PY{l+m+mf}{185.5}\PY{p}{,}\PY{l+m+mf}{190.9}\PY{p}{,}\PY{l+m+mf}{192.8}\PY{p}{,}\PY{l+m+mf}{188.4}\PY{p}{,}\PY{l+m+mf}{187.0}\PY{p}{]}\PY{p}{)}
        \PY{n}{supplier\PYZus{}2} \PY{o}{=} \PY{n}{pd}\PY{o}{.}\PY{n}{Series}\PY{p}{(}\PY{p}{[}\PY{l+m+mf}{158.6}\PY{p}{,}\PY{l+m+mf}{156.4}\PY{p}{,}\PY{l+m+mf}{157.7}\PY{p}{,}\PY{l+m+mf}{154.1}\PY{p}{,}\PY{l+m+mf}{152.3}\PY{p}{,}\PY{l+m+mf}{159.5}\PY{p}{,}\PY{l+m+mf}{158.1}\PY{p}{,}\PY{l+m+mf}{150.9}\PY{p}{,}\PY{l+m+mf}{156.9}\PY{p}{]}\PY{p}{)}
        \PY{n}{supplier\PYZus{}3} \PY{o}{=} \PY{n}{pd}\PY{o}{.}\PY{n}{Series}\PY{p}{(}\PY{p}{[}\PY{l+m+mf}{218.6}\PY{p}{,}\PY{l+m+mf}{208.4}\PY{p}{,}\PY{l+m+mf}{187.1}\PY{p}{,}\PY{l+m+mf}{199.5}\PY{p}{,}\PY{l+m+mf}{202.0}\PY{p}{,}\PY{l+m+mf}{211.1}\PY{p}{,}\PY{l+m+mf}{197.6}\PY{p}{,}\PY{l+m+mf}{204.4}\PY{p}{,}\PY{l+m+mf}{206.8}\PY{p}{]}\PY{p}{)}
        
        \PY{n}{suppliers} \PY{o}{=} \PY{n}{pd}\PY{o}{.}\PY{n}{DataFrame}\PY{p}{(}\PY{p}{\PYZob{}}\PY{l+s}{\PYZdq{}}\PY{l+s}{supplier\PYZus{}1}\PY{l+s}{\PYZdq{}}\PY{p}{:}\PY{n}{supplier\PYZus{}1}\PY{p}{,} 
                                  \PY{l+s}{\PYZdq{}}\PY{l+s}{supplier\PYZus{}2}\PY{l+s}{\PYZdq{}}\PY{p}{:}\PY{n}{supplier\PYZus{}2}\PY{p}{,} 
                                  \PY{l+s}{\PYZdq{}}\PY{l+s}{supplier\PYZus{}3}\PY{l+s}{\PYZdq{}}\PY{p}{:}\PY{n}{supplier\PYZus{}3}\PY{p}{\PYZcb{}}\PY{p}{)}
\end{Verbatim}

    \paragraph{3a. Compute mean and standard deviation for each
supplier}\label{a.-compute-mean-and-standard-deviation-for-each-supplier}

    \begin{Verbatim}[commandchars=\\\{\}]
{\color{incolor}In [{\color{incolor}8}]:} \PY{k}{print} \PY{l+s}{\PYZdq{}}\PY{l+s}{Supplier 1: mean: }\PY{l+s+si}{\PYZpc{}s}\PY{l+s}{, std dev: }\PY{l+s+si}{\PYZpc{}s}\PY{l+s}{\PYZdq{}} \PY{o}{\PYZpc{}} 
                \PY{p}{(}\PY{n}{suppliers}\PY{o}{.}\PY{n}{supplier\PYZus{}1}\PY{o}{.}\PY{n}{mean}\PY{p}{(}\PY{p}{)}\PY{p}{,} \PY{n}{suppliers}\PY{o}{.}\PY{n}{supplier\PYZus{}1}\PY{o}{.}\PY{n}{std}\PY{p}{(}\PY{p}{)}\PY{p}{)}
        \PY{k}{print} \PY{l+s}{\PYZdq{}}\PY{l+s}{Supplier 2: mean: }\PY{l+s+si}{\PYZpc{}s}\PY{l+s}{, std dev: }\PY{l+s+si}{\PYZpc{}s}\PY{l+s}{\PYZdq{}} \PY{o}{\PYZpc{}} 
                \PY{p}{(}\PY{n}{suppliers}\PY{o}{.}\PY{n}{supplier\PYZus{}2}\PY{o}{.}\PY{n}{mean}\PY{p}{(}\PY{p}{)}\PY{p}{,} \PY{n}{suppliers}\PY{o}{.}\PY{n}{supplier\PYZus{}2}\PY{o}{.}\PY{n}{std}\PY{p}{(}\PY{p}{)}\PY{p}{)}
        \PY{k}{print} \PY{l+s}{\PYZdq{}}\PY{l+s}{Supplier 3: mean: }\PY{l+s+si}{\PYZpc{}s}\PY{l+s}{, std dev: }\PY{l+s+si}{\PYZpc{}s}\PY{l+s}{\PYZdq{}} \PY{o}{\PYZpc{}} 
                \PY{p}{(}\PY{n}{suppliers}\PY{o}{.}\PY{n}{supplier\PYZus{}3}\PY{o}{.}\PY{n}{mean}\PY{p}{(}\PY{p}{)}\PY{p}{,} \PY{n}{suppliers}\PY{o}{.}\PY{n}{supplier\PYZus{}3}\PY{o}{.}\PY{n}{std}\PY{p}{(}\PY{p}{)}\PY{p}{)}
\end{Verbatim}

    \begin{Verbatim}[commandchars=\\\{\}]
Supplier 1: mean: 189.011111111, std dev: 3.05223051409
Supplier 2: mean: 156.055555556, std dev: 2.96989524694
Supplier 3: mean: 203.944444444, std dev: 8.95629821845
    \end{Verbatim}

    \paragraph{3b. Plot the data}\label{b.-plot-the-data}

    \begin{Verbatim}[commandchars=\\\{\}]
{\color{incolor}In [{\color{incolor}9}]:} \PY{o}{\PYZpc{}}\PY{k}{matplotlib} inline
        \PY{k+kn}{from} \PY{n+nn}{matplotlib} \PY{k+kn}{import} \PY{n}{pyplot} \PY{k}{as} \PY{n}{plt}
        
        \PY{n}{plt}\PY{o}{.}\PY{n}{hist}\PY{p}{(}\PY{n}{suppliers}\PY{o}{.}\PY{n}{supplier\PYZus{}1}\PY{p}{,} \PY{n}{label}\PY{o}{=}\PY{l+s}{\PYZsq{}}\PY{l+s}{Supplier 1}\PY{l+s}{\PYZsq{}}\PY{p}{,} \PY{n}{normed}\PY{o}{=}\PY{n+nb+bp}{True}\PY{p}{,} \PY{n}{color}\PY{o}{=}\PY{l+s}{\PYZsq{}}\PY{l+s}{g}\PY{l+s}{\PYZsq{}}\PY{p}{)}
        \PY{n}{plt}\PY{o}{.}\PY{n}{hist}\PY{p}{(}\PY{n}{suppliers}\PY{o}{.}\PY{n}{supplier\PYZus{}2}\PY{p}{,} \PY{n}{label}\PY{o}{=}\PY{l+s}{\PYZsq{}}\PY{l+s}{Supplier 2}\PY{l+s}{\PYZsq{}}\PY{p}{,} \PY{n}{normed}\PY{o}{=}\PY{n+nb+bp}{True}\PY{p}{,} \PY{n}{color}\PY{o}{=}\PY{l+s}{\PYZsq{}}\PY{l+s}{b}\PY{l+s}{\PYZsq{}}\PY{p}{)}
        \PY{n}{plt}\PY{o}{.}\PY{n}{hist}\PY{p}{(}\PY{n}{suppliers}\PY{o}{.}\PY{n}{supplier\PYZus{}3}\PY{p}{,} \PY{n}{label}\PY{o}{=}\PY{l+s}{\PYZsq{}}\PY{l+s}{Supplier 3}\PY{l+s}{\PYZsq{}}\PY{p}{,} \PY{n}{normed}\PY{o}{=}\PY{n+nb+bp}{True}\PY{p}{,} \PY{n}{color}\PY{o}{=}\PY{l+s}{\PYZsq{}}\PY{l+s}{r}\PY{l+s}{\PYZsq{}}\PY{p}{)}
        \PY{n}{plt}\PY{o}{.}\PY{n}{title}\PY{p}{(}\PY{l+s}{\PYZdq{}}\PY{l+s}{Supplier sample distances}\PY{l+s}{\PYZdq{}}\PY{p}{)}
        \PY{n}{plt}\PY{o}{.}\PY{n}{xlabel}\PY{p}{(}\PY{l+s}{\PYZdq{}}\PY{l+s}{Sample distance}\PY{l+s}{\PYZdq{}}\PY{p}{)}
        \PY{n}{plt}\PY{o}{.}\PY{n}{ylabel}\PY{p}{(}\PY{l+s}{\PYZdq{}}\PY{l+s}{Frequency}\PY{l+s}{\PYZdq{}}\PY{p}{)}
        \PY{n}{plt}\PY{o}{.}\PY{n}{legend}\PY{p}{(}\PY{p}{)}
        
        \PY{n}{plt}\PY{o}{.}\PY{n}{show}\PY{p}{(}\PY{p}{)}
\end{Verbatim}

    \begin{center}
    \adjustimage{max size={0.9\linewidth}{0.9\paperheight}}{HW2_files/HW2_26_0.png}
    \end{center}
    { \hspace*{\fill} \\}
    
    \paragraph{3c. Which supplier appears to provide material that produces
lenses having power closest to the target
value?}\label{c.-which-supplier-appears-to-provide-material-that-produces-lenses-having-power-closest-to-the-target-value}

    If I understood the question correctly, these values are distances from
a target which means that Supplier 2 is clearly the winner since their
distances are the lowest of the 3. The data suggests that on average,
Supplier 2 is about 156 units away from their target whereas Suppliers 1
and 3 are about 189 and 204 units away from their target, respectively.

    \begin{center}\rule{3in}{0.4pt}\end{center}

\subsubsection{Question 4}\label{question-4}

    \begin{Verbatim}[commandchars=\\\{\}]
{\color{incolor}In [{\color{incolor}10}]:} \PY{n}{df} \PY{o}{=} \PY{n}{pd}\PY{o}{.}\PY{n}{DataFrame}\PY{p}{(}\PY{p}{\PYZob{}}\PY{l+s}{\PYZdq{}}\PY{l+s}{demands}\PY{l+s}{\PYZdq{}}\PY{p}{:} \PY{n}{pd}\PY{o}{.}\PY{n}{Series}\PY{p}{(}\PY{p}{[}\PY{l+m+mi}{28}\PY{p}{,}\PY{l+m+mi}{50}\PY{p}{,}\PY{l+m+mi}{193}\PY{p}{,}\PY{l+m+mi}{65}\PY{p}{,}\PY{l+m+mi}{4}\PY{p}{,}\PY{l+m+mi}{7}\PY{p}{,}\PY{l+m+mi}{147}\PY{p}{,}\PY{l+m+mi}{76}\PY{p}{,}\PY{l+m+mi}{10}\PY{p}{,}\PY{l+m+mi}{0}\PY{p}{,}\PY{l+m+mi}{10}\PY{p}{,}
                                                  \PY{l+m+mi}{84}\PY{p}{,}\PY{l+m+mi}{0}\PY{p}{,}\PY{l+m+mi}{9}\PY{p}{,}\PY{l+m+mi}{1}\PY{p}{,}\PY{l+m+mi}{0}\PY{p}{,}\PY{l+m+mi}{62}\PY{p}{,}\PY{l+m+mi}{26}\PY{p}{,}\PY{l+m+mi}{15}\PY{p}{,}\PY{l+m+mi}{226}\PY{p}{,}\PY{l+m+mi}{54}\PY{p}{,}\PY{l+m+mi}{46}\PY{p}{,}\PY{l+m+mi}{108}\PY{p}{,}
                                                  \PY{l+m+mi}{4}\PY{p}{,}\PY{l+m+mi}{105}\PY{p}{,}\PY{l+m+mi}{40}\PY{p}{,}\PY{l+m+mi}{4}\PY{p}{,}\PY{l+m+mi}{273}\PY{p}{,}\PY{l+m+mi}{184}\PY{p}{,}\PY{l+m+mi}{7}\PY{p}{,}\PY{l+m+mi}{55}\PY{p}{,}\PY{l+m+mi}{41}\PY{p}{,}\PY{l+m+mi}{26}\PY{p}{,}\PY{l+m+mi}{6}\PY{p}{]}\PY{p}{)}\PY{p}{\PYZcb{}}\PY{p}{)}
\end{Verbatim}

    \paragraph{4a. Calculate mean and median for the
data}\label{a.-calculate-mean-and-median-for-the-data}

    \begin{Verbatim}[commandchars=\\\{\}]
{\color{incolor}In [{\color{incolor}11}]:} \PY{n}{mean} \PY{o}{=} \PY{n}{df}\PY{o}{.}\PY{n}{demands}\PY{o}{.}\PY{n}{mean}\PY{p}{(}\PY{p}{)}
         \PY{n}{median} \PY{o}{=} \PY{n}{df}\PY{o}{.}\PY{n}{demands}\PY{o}{.}\PY{n}{median}\PY{p}{(}\PY{p}{)}
         \PY{k}{print} \PY{l+s}{\PYZdq{}}\PY{l+s}{Mean: }\PY{l+s+si}{\PYZpc{}s}\PY{l+s}{\PYZdq{}} \PY{o}{\PYZpc{}} \PY{n}{mean}
         \PY{k}{print} \PY{l+s}{\PYZdq{}}\PY{l+s}{Median: }\PY{l+s+si}{\PYZpc{}s}\PY{l+s}{\PYZdq{}} \PY{o}{\PYZpc{}} \PY{n}{median}
\end{Verbatim}

    \begin{Verbatim}[commandchars=\\\{\}]
Mean: 57.8235294118
Median: 34.0
    \end{Verbatim}

    \paragraph{4b. Which measure appears to best represent the center of the
data?}\label{b.-which-measure-appears-to-best-represent-the-center-of-the-data}

    In this case, I think the median is a better measure of the center of
the data since it is a resistant measure and this data is skewed right
by a few extreme values.

    \paragraph{4c. Calculate range and standard
deviation.}\label{c.-calculate-range-and-standard-deviation.}

    \begin{Verbatim}[commandchars=\\\{\}]
{\color{incolor}In [{\color{incolor}12}]:} \PY{n}{dfRange} \PY{o}{=} \PY{n}{df}\PY{o}{.}\PY{n}{demands}\PY{o}{.}\PY{n}{max}\PY{p}{(}\PY{p}{)} \PY{o}{\PYZhy{}} \PY{n}{df}\PY{o}{.}\PY{n}{demands}\PY{o}{.}\PY{n}{min}\PY{p}{(}\PY{p}{)}
         \PY{n}{stdDev} \PY{o}{=} \PY{n}{df}\PY{o}{.}\PY{n}{demands}\PY{o}{.}\PY{n}{std}\PY{p}{(}\PY{p}{)}
         \PY{k}{print} \PY{l+s}{\PYZdq{}}\PY{l+s}{Range: }\PY{l+s+si}{\PYZpc{}s}\PY{l+s}{\PYZdq{}} \PY{o}{\PYZpc{}} \PY{n}{dfRange}
         \PY{k}{print} \PY{l+s}{\PYZdq{}}\PY{l+s}{Standard dev: }\PY{l+s+si}{\PYZpc{}s}\PY{l+s}{\PYZdq{}} \PY{o}{\PYZpc{}} \PY{n}{stdDev}
\end{Verbatim}

    \begin{Verbatim}[commandchars=\\\{\}]
Range: 273
Standard dev: 70.6873057003
    \end{Verbatim}

    \paragraph{4d. Use range to estimate std dev. How close is the
approximation?}\label{d.-use-range-to-estimate-std-dev.-how-close-is-the-approximation}

    \begin{Verbatim}[commandchars=\\\{\}]
{\color{incolor}In [{\color{incolor}13}]:} \PY{k}{print} \PY{l+s}{\PYZdq{}}\PY{l+s}{Standard dev approximation: }\PY{l+s+si}{\PYZpc{}s}\PY{l+s}{\PYZdq{}} \PY{o}{\PYZpc{}} \PY{p}{(}\PY{n}{dfRange} \PY{o}{/} \PY{l+m+mf}{4.}\PY{p}{)}
\end{Verbatim}

    \begin{Verbatim}[commandchars=\\\{\}]
Standard dev approximation: 68.25
    \end{Verbatim}

    The approximation is actually pretty close to the true value. It's off
by about 2.5 units (about 4\%)

    \paragraph{4e. Construct intervals for
$\bar{y} \pm 1s, \bar{y} \pm 2s, \bar{y} \pm 3s$, count number of data
falling in each interval, convert to percentages and compare with
Empirical
Rule.}\label{e.-construct-intervals-for-bary-pm-1s-bary-pm-2s-bary-pm-3s-count-number-of-data-falling-in-each-interval-convert-to-percentages-and-compare-with-empirical-rule.}

    \begin{Verbatim}[commandchars=\\\{\}]
{\color{incolor}In [{\color{incolor}14}]:} \PY{n}{output} \PY{o}{=} \PY{l+s}{\PYZdq{}}\PY{l+s}{[\PYZob{}0[0]\PYZcb{}, \PYZob{}0[1]\PYZcb{}]: \PYZob{}1\PYZcb{} items =\PYZgt{} \PYZob{}2:.2f\PYZcb{}}\PY{l+s}{\PYZpc{}}\PY{l+s}{\PYZdq{}} \PY{c}{\PYZsh{}output format string}
         
         \PY{n}{firstInt} \PY{o}{=} \PY{p}{(}\PY{n}{mean} \PY{o}{\PYZhy{}} \PY{n}{stdDev}\PY{p}{,} \PY{n}{mean} \PY{o}{+} \PY{n}{stdDev}\PY{p}{)}
         \PY{n}{firstIntValues} \PY{o}{=} \PY{n}{df}\PY{p}{[}\PY{p}{(}\PY{n}{df}\PY{o}{.}\PY{n}{demands} \PY{o}{\PYZgt{}}\PY{o}{=} \PY{n}{firstInt}\PY{p}{[}\PY{l+m+mi}{0}\PY{p}{]}\PY{p}{)} \PY{o}{\PYZam{}} \PY{p}{(}\PY{n}{df}\PY{o}{.}\PY{n}{demands} \PY{o}{\PYZlt{}}\PY{o}{=} \PY{n}{firstInt}\PY{p}{[}\PY{l+m+mi}{1}\PY{p}{]}\PY{p}{)}\PY{p}{]}
         \PY{n}{firstCount} \PY{o}{=} \PY{n+nb}{len}\PY{p}{(}\PY{n}{firstIntValues}\PY{p}{)}
         \PY{n}{firstPercentage} \PY{o}{=} \PY{l+m+mf}{100.0} \PY{o}{*} \PY{n}{firstCount} \PY{o}{/} \PY{n+nb}{len}\PY{p}{(}\PY{n}{df}\PY{p}{)}
         \PY{k}{print} \PY{n}{output}\PY{o}{.}\PY{n}{format}\PY{p}{(}\PY{n}{firstInt}\PY{p}{,} \PY{n}{firstCount}\PY{p}{,} \PY{n}{firstPercentage}\PY{p}{)}
         
         \PY{n}{secondInt} \PY{o}{=} \PY{p}{(}\PY{n}{mean} \PY{o}{\PYZhy{}} \PY{l+m+mi}{2}\PY{o}{*}\PY{n}{stdDev}\PY{p}{,} \PY{n}{mean} \PY{o}{+} \PY{l+m+mi}{2}\PY{o}{*}\PY{n}{stdDev}\PY{p}{)}
         \PY{n}{secondIntValues} \PY{o}{=} \PY{n}{df}\PY{p}{[}\PY{p}{(}\PY{n}{df}\PY{o}{.}\PY{n}{demands} \PY{o}{\PYZgt{}}\PY{o}{=} \PY{n}{secondInt}\PY{p}{[}\PY{l+m+mi}{0}\PY{p}{]}\PY{p}{)} \PY{o}{\PYZam{}} \PY{p}{(}\PY{n}{df}\PY{o}{.}\PY{n}{demands} \PY{o}{\PYZlt{}}\PY{o}{=} \PY{n}{secondInt}\PY{p}{[}\PY{l+m+mi}{1}\PY{p}{]}\PY{p}{)}\PY{p}{]}
         \PY{n}{secondCount} \PY{o}{=} \PY{n+nb}{len}\PY{p}{(}\PY{n}{secondIntValues}\PY{p}{)}
         \PY{n}{secondPercentage} \PY{o}{=} \PY{l+m+mf}{100.0} \PY{o}{*} \PY{n}{secondCount} \PY{o}{/} \PY{n+nb}{len}\PY{p}{(}\PY{n}{df}\PY{p}{)}
         \PY{k}{print} \PY{n}{output}\PY{o}{.}\PY{n}{format}\PY{p}{(}\PY{n}{secondInt}\PY{p}{,} \PY{n}{secondCount}\PY{p}{,} \PY{n}{secondPercentage}\PY{p}{)}
         
         \PY{n}{thirdInt} \PY{o}{=} \PY{p}{(}\PY{n}{mean} \PY{o}{\PYZhy{}} \PY{l+m+mi}{3}\PY{o}{*}\PY{n}{stdDev}\PY{p}{,} \PY{n}{mean} \PY{o}{+} \PY{l+m+mi}{3}\PY{o}{*}\PY{n}{stdDev}\PY{p}{)}
         \PY{n}{thirdIntValues} \PY{o}{=} \PY{n}{df}\PY{p}{[}\PY{p}{(}\PY{n}{df}\PY{o}{.}\PY{n}{demands} \PY{o}{\PYZgt{}}\PY{o}{=} \PY{n}{thirdInt}\PY{p}{[}\PY{l+m+mi}{0}\PY{p}{]}\PY{p}{)} \PY{o}{\PYZam{}} \PY{p}{(}\PY{n}{df}\PY{o}{.}\PY{n}{demands} \PY{o}{\PYZlt{}}\PY{o}{=} \PY{n}{thirdInt}\PY{p}{[}\PY{l+m+mi}{1}\PY{p}{]}\PY{p}{)}\PY{p}{]}
         \PY{n}{thirdCount} \PY{o}{=} \PY{n+nb}{len}\PY{p}{(}\PY{n}{thirdIntValues}\PY{p}{)}
         \PY{n}{thirdPercentage} \PY{o}{=} \PY{l+m+mf}{100.0} \PY{o}{*} \PY{n}{thirdCount} \PY{o}{/} \PY{n+nb}{len}\PY{p}{(}\PY{n}{df}\PY{p}{)}
         \PY{k}{print} \PY{n}{output}\PY{o}{.}\PY{n}{format}\PY{p}{(}\PY{n}{thirdInt}\PY{p}{,} \PY{n}{thirdCount}\PY{p}{,} \PY{n}{thirdPercentage}\PY{p}{)}
\end{Verbatim}

    \begin{Verbatim}[commandchars=\\\{\}]
[-12.8637762885, 128.510835112]: 29 items => 85.29\%
[-83.5510819888, 199.198140812]: 32 items => 94.12\%
[-154.238387689, 269.885446513]: 33 items => 97.06\%
    \end{Verbatim}

    \paragraph{4f. Why do you think the Empirical Rule and your percentages
do not match
well?}\label{f.-why-do-you-think-the-empirical-rule-and-your-percentages-do-not-match-well}

    \begin{Verbatim}[commandchars=\\\{\}]
{\color{incolor}In [{\color{incolor}15}]:} \PY{n}{df}\PY{o}{.}\PY{n}{plot}\PY{p}{(}\PY{n}{kind}\PY{o}{=}\PY{l+s}{\PYZsq{}}\PY{l+s}{hist}\PY{l+s}{\PYZsq{}}\PY{p}{)}
\end{Verbatim}

            \begin{Verbatim}[commandchars=\\\{\}]
{\color{outcolor}Out[{\color{outcolor}15}]:} <matplotlib.axes.\_subplots.AxesSubplot at 0x7fb25b18ac50>
\end{Verbatim}
        
    \begin{center}
    \adjustimage{max size={0.9\linewidth}{0.9\paperheight}}{HW2_files/HW2_43_1.png}
    \end{center}
    { \hspace*{\fill} \\}
    
    The empirical rule percentages did not match well with the percentages
above because, as the histogram of the data shows, it is not evenly
distributed. It has a large concentration of values on one side of the
distribution and a few extreme outliers to the right. For this reason,
over 85\% of the values were within one standard deviation of the mean.

    \begin{center}\rule{3in}{0.4pt}\end{center}

    \newpage 
\subsubsection{Question 5}\label{question-5}

    \begin{Verbatim}[commandchars=\\\{\}]
{\color{incolor}In [{\color{incolor}16}]:} \PY{n}{df}\PY{o}{.}\PY{n}{plot}\PY{p}{(}\PY{n}{kind}\PY{o}{=}\PY{l+s}{\PYZsq{}}\PY{l+s}{box}\PY{l+s}{\PYZsq{}}\PY{p}{,} \PY{n}{vert}\PY{o}{=}\PY{n+nb+bp}{False}\PY{p}{,} \PY{n}{xlim}\PY{o}{=}\PY{p}{(}\PY{o}{\PYZhy{}}\PY{l+m+mi}{50}\PY{p}{,}\PY{l+m+mi}{300}\PY{p}{)}\PY{p}{)}
\end{Verbatim}

            \begin{Verbatim}[commandchars=\\\{\}]
{\color{outcolor}Out[{\color{outcolor}16}]:} <matplotlib.axes.\_subplots.AxesSubplot at 0x7fb25898a210>
\end{Verbatim}
        
    \begin{center}
    \adjustimage{max size={0.9\linewidth}{0.9\paperheight}}{HW2_files/HW2_46_1.png}
    \end{center}
    { \hspace*{\fill} \\}
    
    \paragraph{5a. Is the data left-skewed, symmetric or
right-skewed?}\label{a.-is-the-data-left-skewed-symmetric-or-right-skewed}

    The data is right-skewed as shown in Lesson 2.3. The box is longer
towards the right, and the line from the 3rd quartile to the upper
adjacent value is longer.

    \paragraph{5b. What are the outliers?}\label{b.-what-are-the-outliers}

    The potential outliers are the values greater than 147: {[} 184, 193,
226, 273 {]}

    \paragraph{5c. Is the median closer to the lower quartile or the upper
quartile? Does that indicate that the density of data between first
quartile and the median is higher than the density of data between the
median and the third
quartile?}\label{c.-is-the-median-closer-to-the-lower-quartile-or-the-upper-quartile-does-that-indicate-that-the-density-of-data-between-first-quartile-and-the-median-is-higher-than-the-density-of-data-between-the-median-and-the-third-quartile}

    For this data, the median is closer to the lower quartile. In this
scenario, there is a difference in density. There are many more points
between the first quartile and the median than between the median and
the third quartile. We can see that corroborated by the histogram above.
However, I don't think that the median being closer to the lower
quartile necessarily indicates that. It could also be a result of how
sparse the data points are, especially with fewer samples.

    \begin{center}\rule{3in}{0.4pt}\end{center}

\subsubsection{Question 6}\label{question-6}

    \begin{Verbatim}[commandchars=\\\{\}]
{\color{incolor}In [{\color{incolor}17}]:} \PY{n}{paperTowelData} \PY{o}{=} \PY{n}{pd}\PY{o}{.}\PY{n}{read\PYZus{}csv}\PY{p}{(}\PY{l+s}{\PYZsq{}}\PY{l+s}{hw2\PYZus{}paperTowelData.csv}\PY{l+s}{\PYZsq{}}\PY{p}{,} \PY{n}{index\PYZus{}col}\PY{o}{=}\PY{n+nb+bp}{None}\PY{p}{)}
         \PY{n}{pricePerRoll} \PY{o}{=} \PY{n}{paperTowelData}\PY{p}{[}\PY{l+s}{\PYZsq{}}\PY{l+s}{Price per Roll}\PY{l+s}{\PYZsq{}}\PY{p}{]}
         \PY{n}{pricePerSheet} \PY{o}{=} \PY{n}{paperTowelData}\PY{p}{[}\PY{l+s}{\PYZsq{}}\PY{l+s}{Price per Sheet}\PY{l+s}{\PYZsq{}}\PY{p}{]}
\end{Verbatim}

    \paragraph{6a. Compute standard deviation for both price per roll and
price per
sheet.}\label{a.-compute-standard-deviation-for-both-price-per-roll-and-price-per-sheet.}

    \begin{Verbatim}[commandchars=\\\{\}]
{\color{incolor}In [{\color{incolor}18}]:} \PY{k}{print} \PY{l+s}{\PYZdq{}}\PY{l+s}{Price per roll std dev: }\PY{l+s+si}{\PYZpc{}s}\PY{l+s}{\PYZdq{}} \PY{o}{\PYZpc{}} \PY{n}{pricePerRoll}\PY{o}{.}\PY{n}{std}\PY{p}{(}\PY{p}{)}
         \PY{k}{print} \PY{l+s}{\PYZdq{}}\PY{l+s}{Price per sheet std dev: }\PY{l+s+si}{\PYZpc{}s}\PY{l+s}{\PYZdq{}} \PY{o}{\PYZpc{}} \PY{n}{pricePerSheet}\PY{o}{.}\PY{n}{std}\PY{p}{(}\PY{p}{)}
\end{Verbatim}

    \begin{Verbatim}[commandchars=\\\{\}]
Price per roll std dev: 0.423304589347
Price per sheet std dev: 0.00553029698864
    \end{Verbatim}

    \paragraph{6b. Which is more variable, price per roll or price per
sheet? Should you use $s$ (standard deviation) or $CV$ (Coefficient of
Variation)?}\label{b.-which-is-more-variable-price-per-roll-or-price-per-sheet-should-you-use-s-standard-deviation-or-cv-coefficient-of-variation}

    \begin{Verbatim}[commandchars=\\\{\}]
{\color{incolor}In [{\color{incolor}19}]:} \PY{n}{perRollCV} \PY{o}{=} \PY{n}{pricePerRoll}\PY{o}{.}\PY{n}{std}\PY{p}{(}\PY{p}{)} \PY{o}{/} \PY{n}{pricePerRoll}\PY{o}{.}\PY{n}{mean}\PY{p}{(}\PY{p}{)}
         \PY{n}{perSheetCV} \PY{o}{=} \PY{n}{pricePerSheet}\PY{o}{.}\PY{n}{std}\PY{p}{(}\PY{p}{)} \PY{o}{/} \PY{n}{pricePerSheet}\PY{o}{.}\PY{n}{mean}\PY{p}{(}\PY{p}{)}
         
         \PY{k}{print} \PY{l+s}{\PYZdq{}}\PY{l+s}{Price per roll CV: }\PY{l+s+si}{\PYZpc{}.2f}\PY{l+s+si}{\PYZpc{}\PYZpc{}}\PY{l+s}{\PYZdq{}} \PY{o}{\PYZpc{}} \PY{p}{(}\PY{n}{perRollCV} \PY{o}{*} \PY{l+m+mi}{100}\PY{p}{)}
         \PY{k}{print} \PY{l+s}{\PYZdq{}}\PY{l+s}{Price per sheet CV: }\PY{l+s+si}{\PYZpc{}.2f}\PY{l+s+si}{\PYZpc{}\PYZpc{}}\PY{l+s}{\PYZdq{}} \PY{o}{\PYZpc{}} \PY{p}{(}\PY{n}{perSheetCV} \PY{o}{*} \PY{l+m+mi}{100}\PY{p}{)}
\end{Verbatim}

    \begin{Verbatim}[commandchars=\\\{\}]
Price per roll CV: 46.03\%
Price per sheet CV: 48.78\%
    \end{Verbatim}

    Although the price per roll has a higher standard deviation, the price
per sheet has a higher CV, which indicates that the price per sheet has
greater variability relative to the magnitude of it's mean.

    \paragraph{6c. Do you think that IQR is a good measure to compare the
variability of these two
variables?}\label{c.-do-you-think-that-iqr-is-a-good-measure-to-compare-the-variability-of-these-two-variables}

    I think that IQR would be just as poor as standard deviation for
comparing variability since both of those measures are in the same units
as the data. That makes it difficult to compare across data sets.

    \begin{center}\rule{3in}{0.4pt}\end{center}

\subsubsection{Question 7}\label{question-7}

    \begin{Verbatim}[commandchars=\\\{\}]
{\color{incolor}In [{\color{incolor}20}]:} \PY{n}{singerData} \PY{o}{=} \PY{n}{pd}\PY{o}{.}\PY{n}{read\PYZus{}csv}\PY{p}{(}\PY{l+s}{\PYZsq{}}\PY{l+s}{hw2\PYZus{}singers.csv}\PY{l+s}{\PYZsq{}}\PY{p}{)}
         \PY{n}{basses} \PY{o}{=} \PY{n}{singerData}\PY{p}{[}\PY{l+s}{\PYZsq{}}\PY{l+s}{Bass}\PY{l+s}{\PYZsq{}}\PY{p}{]}
         \PY{n}{sopranos} \PY{o}{=} \PY{n}{singerData}\PY{p}{[}\PY{l+s}{\PYZsq{}}\PY{l+s}{Soprano}\PY{l+s}{\PYZsq{}}\PY{p}{]}
         \PY{n}{altos} \PY{o}{=} \PY{n}{singerData}\PY{p}{[}\PY{l+s}{\PYZsq{}}\PY{l+s}{Alto}\PY{l+s}{\PYZsq{}}\PY{p}{]}
         \PY{n}{tenors} \PY{o}{=} \PY{n}{singerData}\PY{p}{[}\PY{l+s}{\PYZsq{}}\PY{l+s}{Tenor}\PY{l+s}{\PYZsq{}}\PY{p}{]}
\end{Verbatim}

    \paragraph{7a. Find descriptive statistics for each type of singer. For
each case, does the approximate value of $s$ give a good estimate of
$s$?}\label{a.-find-descriptive-statistics-for-each-type-of-singer.-for-each-case-does-the-approximate-value-of-s-give-a-good-estimate-of-s}

    \begin{Verbatim}[commandchars=\\\{\}]
{\color{incolor}In [{\color{incolor}21}]:} \PY{k}{print} \PY{l+s}{\PYZdq{}}\PY{l+s}{Bass singers: }\PY{l+s+se}{\PYZbs{}n}\PY{l+s+si}{\PYZpc{}s}\PY{l+s+se}{\PYZbs{}n}\PY{l+s}{\PYZdq{}} \PY{o}{\PYZpc{}} \PY{n}{basses}\PY{o}{.}\PY{n}{describe}\PY{p}{(}\PY{p}{)}
         \PY{k}{print} \PY{l+s}{\PYZdq{}}\PY{l+s}{Alto singers: }\PY{l+s+se}{\PYZbs{}n}\PY{l+s+si}{\PYZpc{}s}\PY{l+s+se}{\PYZbs{}n}\PY{l+s}{\PYZdq{}} \PY{o}{\PYZpc{}} \PY{n}{altos}\PY{o}{.}\PY{n}{describe}\PY{p}{(}\PY{p}{)}
         \PY{k}{print} \PY{l+s}{\PYZdq{}}\PY{l+s}{Soprano singers: }\PY{l+s+se}{\PYZbs{}n}\PY{l+s+si}{\PYZpc{}s}\PY{l+s+se}{\PYZbs{}n}\PY{l+s}{\PYZdq{}} \PY{o}{\PYZpc{}} \PY{n}{sopranos}\PY{o}{.}\PY{n}{describe}\PY{p}{(}\PY{p}{)}
         \PY{k}{print} \PY{l+s}{\PYZdq{}}\PY{l+s}{Tenor singers: }\PY{l+s+se}{\PYZbs{}n}\PY{l+s+si}{\PYZpc{}s}\PY{l+s+se}{\PYZbs{}n}\PY{l+s}{\PYZdq{}} \PY{o}{\PYZpc{}} \PY{n}{tenors}\PY{o}{.}\PY{n}{describe}\PY{p}{(}\PY{p}{)}
\end{Verbatim}

    \begin{Verbatim}[commandchars=\\\{\}]
Bass singers: 
count    36.000000
mean     70.750000
std       2.430461
min      66.000000
25\%      69.000000
50\%      71.000000
75\%      72.000000
max      75.000000
Name: Bass, dtype: float64

Alto singers: 
count    35.000000
mean     64.885714
std       2.794653
min      60.000000
25\%      63.000000
50\%      65.000000
75\%      66.500000
max      72.000000
Name: Alto, dtype: float64

Soprano singers: 
count    39.000000
mean     64.717949
std       2.459661
min      60.000000
25\%      63.000000
50\%      65.000000
75\%      66.000000
max      72.000000
Name: Soprano, dtype: float64

Tenor singers: 
count    20.000000
mean     69.150000
std       3.216323
min      64.000000
25\%      66.750000
50\%      68.500000
75\%      71.250000
max      76.000000
Name: Tenor, dtype: float64
    \end{Verbatim}

    \begin{Verbatim}[commandchars=\\\{\}]
{\color{incolor}In [{\color{incolor}22}]:} \PY{k}{print} \PY{l+s}{\PYZdq{}}\PY{l+s}{Bass approx s: }\PY{l+s+si}{\PYZpc{}s}\PY{l+s}{\PYZdq{}} \PY{o}{\PYZpc{}} \PY{p}{(}\PY{p}{(}\PY{n}{basses}\PY{o}{.}\PY{n}{max}\PY{p}{(}\PY{p}{)} \PY{o}{\PYZhy{}} \PY{n}{basses}\PY{o}{.}\PY{n}{min}\PY{p}{(}\PY{p}{)}\PY{p}{)} \PY{o}{/} \PY{l+m+mf}{4.}\PY{p}{)}
         \PY{k}{print} \PY{l+s}{\PYZdq{}}\PY{l+s}{Alot approx s: }\PY{l+s+si}{\PYZpc{}s}\PY{l+s}{\PYZdq{}} \PY{o}{\PYZpc{}} \PY{p}{(}\PY{p}{(}\PY{n}{altos}\PY{o}{.}\PY{n}{max}\PY{p}{(}\PY{p}{)} \PY{o}{\PYZhy{}} \PY{n}{altos}\PY{o}{.}\PY{n}{min}\PY{p}{(}\PY{p}{)}\PY{p}{)} \PY{o}{/} \PY{l+m+mf}{4.}\PY{p}{)}
         \PY{k}{print} \PY{l+s}{\PYZdq{}}\PY{l+s}{Soprano approx s: }\PY{l+s+si}{\PYZpc{}s}\PY{l+s}{\PYZdq{}} \PY{o}{\PYZpc{}} \PY{p}{(}\PY{p}{(}\PY{n}{sopranos}\PY{o}{.}\PY{n}{max}\PY{p}{(}\PY{p}{)} \PY{o}{\PYZhy{}} \PY{n}{sopranos}\PY{o}{.}\PY{n}{min}\PY{p}{(}\PY{p}{)}\PY{p}{)} \PY{o}{/} \PY{l+m+mf}{4.}\PY{p}{)}
         \PY{k}{print} \PY{l+s}{\PYZdq{}}\PY{l+s}{Tenor approx s: }\PY{l+s+si}{\PYZpc{}s}\PY{l+s}{\PYZdq{}} \PY{o}{\PYZpc{}} \PY{p}{(}\PY{p}{(}\PY{n}{tenors}\PY{o}{.}\PY{n}{max}\PY{p}{(}\PY{p}{)} \PY{o}{\PYZhy{}} \PY{n}{tenors}\PY{o}{.}\PY{n}{min}\PY{p}{(}\PY{p}{)}\PY{p}{)} \PY{o}{/} \PY{l+m+mf}{4.}\PY{p}{)}
\end{Verbatim}

    \begin{Verbatim}[commandchars=\\\{\}]
Bass approx s: 2.25
Alot approx s: 3.0
Soprano approx s: 3.0
Tenor approx s: 3.0
    \end{Verbatim}

    Comparing the approximate values of $s$ to the actual values of $s$,
they appear to be fair estimates. I think they are close enough to be
useful for getting an idea of the variability of the data sets. The
soprano estimate seems to be the most off by about 22\%.

    \paragraph{7b. Draw boxplots side by side for these cases. Comment on
the central tendency and the dispersion for the four types of
singers.}\label{b.-draw-boxplots-side-by-side-for-these-cases.-comment-on-the-central-tendency-and-the-dispersion-for-the-four-types-of-singers.}

    \begin{Verbatim}[commandchars=\\\{\}]
{\color{incolor}In [{\color{incolor}23}]:} \PY{n}{singerData}\PY{o}{.}\PY{n}{plot}\PY{p}{(}\PY{n}{kind}\PY{o}{=}\PY{l+s}{\PYZsq{}}\PY{l+s}{box}\PY{l+s}{\PYZsq{}}\PY{p}{,} \PY{n}{vert}\PY{o}{=}\PY{n+nb+bp}{False}\PY{p}{,} \PY{n}{xlim}\PY{o}{=}\PY{p}{(}\PY{l+m+mi}{55}\PY{p}{,}\PY{l+m+mi}{80}\PY{p}{)}\PY{p}{)}
\end{Verbatim}

            \begin{Verbatim}[commandchars=\\\{\}]
{\color{outcolor}Out[{\color{outcolor}23}]:} <matplotlib.axes.\_subplots.AxesSubplot at 0x7fb2588fbad0>
\end{Verbatim}
        
    \begin{center}
    \adjustimage{max size={0.9\linewidth}{0.9\paperheight}}{HW2_files/HW2_69_1.png}
    \end{center}
    { \hspace*{\fill} \\}
    
    With regard to central tendency, the means are very close to the
medians. With the exception of the tenor data (which looks a little
skewed right), the data looks very evenly distributed.


    % Add a bibliography block to the postdoc
    
    
    
    \end{document}
